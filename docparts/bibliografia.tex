\begin{thebibliography}{10}%jak ktoś ma więcej książek, to niech wpisze większą liczbę

% \bibitem[numerek]{referencja} Autor, \emph{Tytuł}, Wydawnictwo, rok, strony
% cytowanie: \cite{referencja1, referencja 2,...}

\bibitem[1]{bib:raportKoszty}Knight Frank Sp. z o.o., \emph{Koszty i Opłaty Eksploatacyjne w Budynkach Biurowych 2011-2016} [dostęp 29 XII 2017] 
https://kfcontent.blob.core.windows.net/research/1291/documents/pl/koszty-i-oplaty-eksploatacyjne-w-budynkach-biurowych-2011-2017-4775.pdf

\bibitem[1]{bib:AkkaVsOrleans}Dr R. Kuhn, \emph{Orleans and Akka Actors: A Comparison} [dostęp: 07 XII 2017],
https://github.com/akka/akka-meta/blob/master/ComparisonWithOrleans.md

\bibitem[2]{bib:Orleans2} \emph{[Orleans Github Issue] .Net Core Compatibility} [dostęp: 20 X 2017],
https://github.com/dotnet/orleans/issues/2145

\bibitem[3]{bib:AkkaNoAsk} Bartosz Sypytkowski, \emph{Unable to use await while message processing} [dostęp: 17 X 2017], 
https://github.com/akkadotnet/akka.net/issues/1292\#issuecomment-137730065

% \bibitem[2]{Ktos} A. Aaaaa, \emph{Tytuł}, Wydawnictwo, rok, strona-strona.
% \bibitem[3]{Innyktos} J. Bobkowski, S. Dobkowski, \emph{Blebleble}, Magazyn nr, rok, strony.
% \bibitem[4]{B} C. Brink, \emph{Power structures}, Algebra Universalis 30(2), 1993, 177-216.
% \bibitem[0]{H} F. Burris, H. P. Sankappanavar, \emph{A Course of Universal Algebra}, Springer-Verlag, New York, 1981.

\end{thebibliography}