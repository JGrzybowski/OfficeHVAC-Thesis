\section{Wnioski}
Systemy agentowe, a w szczególności systemy aktorów są wygodnym narzędziem do zarządzania siecią urządzeń. 
W perspektywie coraz większej popularności internetu rzeczy (IoT - Internet of Things) można spodziewać się też co raz częstszego wykorzystania tych rozwiązań w zastosowaniach już nie tylko przemysłowych, jak automatyzacja fabryk, ale i konsumenckich jak automatyka biur lub domów.  

Wykorzystanie tradycyjnych technik znanych z programowania obiektowego, takich jak zdarzenia, w systemie aktorów nie jest możliwe ze względu na zasady przydzielania wątków poszczególnym aktorom. Również 
korzystanie z kontenerów do wstrzykiwania zależności wewnątrz aktorów może prowadzić do dzielenia stanu przez aktorów, co jest sprzeczne z założeniami Akka.net i samego modelu aktorów.

\section{Propozycje dalszego rozwoju systemu}
\subsection*{Dodanie nowych parametrów pomieszczeń}
Powszechnie dostępne sensory temperatury często posiadają również czujnik wilgotności, która wpływa na to, jak szybko nagrzewa się powietrze. Monitorowanie wilgotności pozwalałoby na dokładniejsze wyliczenia zmian temperatury. Zbyt wysoka wilgotność mogłaby być czynnikiem wyzwalającym wietrzenie, aby osobom w pomieszczeniu nie był zbyt duszno.

\subsection*{Udoskonalenie strategii uruchamiania urządzeń}
Zaproponowana strategia uruchamiania urządzeń zakłada uruchamianie wszystkich urządzeń w pomieszczeniu w tym samym trybie na ten sam okres czasu. Być może opracowanie strategii gdzie pewne urządzenia pełnią rolę pomocniczą przez część czasu przygotowywania pomieszczenia dawałoby dodatkowe oszczędności.

\subsection*{Opracowanie nowych źródeł danych}
Przy omawianiu źródeł danych wspomniane zostały alternatywy do kalendarzy firmowych. Przykładowo lokalizacja użytkowników pozwoli zaoszczędzić energię elektryczną, gdy pracownik firmy tkwi przez godzinę w korku i spóźnia się do swojego biura. 

Lokalizacja wewnątrz samego budynku też wydaje się dobrym źródłem informacji o wykorzystaniu pomieszczeń, lecz może okazać się zbyt dynamiczna, aby móc ją wykorzystać w planowaniu.

\subsection*{Douczanie modelu na podstawie danych z systemu}
Pomieszczenia w biurowcach różnią się rozmiarami, kształtem oraz rozmieszczeniem urządzeń HVAC. Powoduje to, że każde z nich ogrzewa się inaczej. Jeżeli dodać do aktora kontrolera moduł poprawiający model temperatury dla danego pomieszczenia, możemy jeszcze bardziej zoptymalizować koszty energii elektrycznej.

\subsection*{Wykorzystanie osobistych preferencji użytkowników}
Jedną z trudniejszych rzeczy dla użytkowników korzystających z proponowanego systemu może być określenie z wyprzedzeniem jaka temperatura będzie dla nich komfortowa.
Zbudowanie modułu w którym byłaby przechowywana informacja o preferowanych temperaturach pozwoliłoby na sugerowanie temperatury podczas spotkania. 