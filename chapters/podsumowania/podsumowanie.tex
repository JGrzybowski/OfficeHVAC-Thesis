\section{Podsumowanie}
Proponowane rozwiązanie pozwala ograniczyć zużycie energii elektrycznej, zapewniając użytkownikom ten sam poziomu komfortu co sterowanie ręczne. 

System jest w stanie reagować na zmiany w wykorzystaniu pomieszczeń, zachodzące przy typowym wykorzystaniu kalendarzy firmowych. Przekłada się to na brak potrzeby szkolenia użytkowników z obsługi nowego systemu HVAC. 

Opracowany uproszczony model jest wystarczający w przypadku gdy mamy do czynienia z niewielką liczbą urządzeń HVAC, gdzie zysk z doprecyzowania modelu nie byłby współmierny do kosztów przygotowania nowego modelu. W przypadku większych sieci urządzeń udoskonalenie modeli predykcji zmiany temperatury, pozwoli na jeszcze większe oszczędności. 

System jest przystosowany do dalszej rozbudowy nie tylko podwzględem poprawy modeli, ale i wymiany sensorów na zbierające więcej danych takich jak wilgotoność czy ilość tlenu i dwutlenku węgla w powietrzu.
Pozwoli to w przyszłości jeszcze bardziej dbać o komfort użytkowników wynajmowanych pomieszczeń.
