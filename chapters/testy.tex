\chapter{Testy}
\section{Testy w Akka.net}
Podstawowymi testami jakie można wykonać są testy jednostkowe posczególnych aktorów.
W Akka.net testy te pisze się za pomocą specjalnej paczki Akka.TestKit wystawiającej zestaw narzędzi do testowania aktorów.

Testy najczęściej mają podobną strukturę:
\begin{enumerate}
    \item Stworzenie aktora.
    \item Przesłanie mu wiadomości inicjalizujących (jeżeli jest taka potrzeba.)
    \item Wysłanie wiadomości uruchamiającej testowane zachowane aktora.
    \item Odczytanie zwróconej w.iadomości i sprawdzenie czy wartości wewnątrz zgadzają się z oczekiwaniami
\end{enumerate}
Istotne jest zauważenie różnicy w stosunku do tradycyjnych testów jednostkowych. Nie sprawdza się stanu wewnętrznego aktora, lecz jego reakcje na wiadomości.

\section{Pułapki związane z testowaniem aktorów}
\subsection{Sprawdzanie stanu wewnętrznego aktora}
Mimo tego, że Akka.TestKit udostępnia metody pozwalające na sprawdzenie stanu wewnętrznego aktora, nie jest to zalecane podejście. 
Prowadzi do sytuacji, gdzie testy zależą od wewnętrznej implementacji. 
Tak jak w tradycyjnych testach jednostkowych testuje się implementacje interfejsów, tak w Akka.net testuje się zachowania na poszczególne obsługiwane wiadomości.

\subsection{Używanie metody \lstinline{Ask}}
\lstinline{Ask<T>} jest metodą pozwalającą oczekiwać na wiadomość zwrotną z systemu aktorów danego typu \lstinline{T}.
Jest ona częścią wewnętrznych mechanizmów Akka.net i sami twórcy nie zalecają jej używania wewnątrz pisanych aktorów. 
\lstinline{Ask<T>} nie pozwala na zmianę wątku co w przypadku testów doprowadza do sytuacji, gdzie test zwraca błąd związany z przekroczeniem czasu oczekiwania na wiadomość zwrotną. Zamiast tego, w testach powinna występować metoda \lstinline{ExpectMsg<T>}, która zwraca pierwszą wiadomość typu \lstinline{T}, która przyjdzie do aktora testowego dostępnego w klasie \lstinline{TestKit}.  

\section{Scenariusze testowe}
Poza testami jednostkowymi, powstały też testy integracyjne łączące w jednym teście większość stworzonych aktorów. Przykładowy test z całym scenariuszem:
\begin{lstlisting}
    Ustawienie początkowej godziny na 7:30
    Stworzenie statusu startowego (25st.C pokój o kubaturze 72m^3)
    Dodanie dostępnych urządzeń
    Stworzenie aktora pomieszczenia

    Dodanie wydarzenia w kalendarzu na 10:30 kończącego się o 12:00 z wymaganą temperaturą 21st.C
    
    Przestawienie zegara na 8:30 
    Dodanie wydarzenia w kalendarzu na 09:00 kończącego się o 10:00 z wymaganą temperaturą 18st.C

    Przestawienie zegara na 9:00
    Sprawdzenie czy w pokoju panuje temperatura 18st.C +- 0.5st.C
    
    Przestawienie zegara na 10:00
    Sprawdzenie czy w pokoju panuje temperatura 18st.C +- 0.5st.C
    
    Przestawienie zegara na 10:30 
    Sprawdzenie czy w pokoju panuje temperatura 21st.C +- 0.5st.C

    Przestawianie zegara na 12:00
    Sprawdzenie czy wszystkie urządzenia w pokoju
     są wyłączone, jako że nie ma przewidzianych 
     innych spotkań w tym pokoju.
\end{lstlisting}