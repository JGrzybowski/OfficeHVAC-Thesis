\chapter{Testy}
\section{Testy w Akka.net}
Podstawowymi testami jakie można wykonać są testy jednostkowe posczególnych aktorów.
W Akka.net testy te pisze się za pomocą specjalnej paczki Akka.TestKit wystawiającej zestaw narzędzi do testowania aktorów.

Testy najczęściej mają podobną strukturę:
\begin{enumerate}
    \item Stworzenie aktora.
    \item Przesłanie mu wiadomości inicjalizujących (jeżeli jest taka potrzeba.)
    \item Wysłanie wiadomości uruchamiającej testowane zachowane aktora.
    \item Odczytanie zwróconej w.iadomości i sprawdzenie czy wartości wewnątrz zgadzają się z oczekiwaniami
\end{enumerate}
Istotne jest zauważenie różnicy w stosunku do tradycyjnych testów jednostkowych. Nie sprawdza się stanu wewnętrznego aktora, lecz jego reakcje na wiadomości.

\section{Pułapki związane z testowaniem aktorów}
\subsection{Sprawdzanie stanu wewnętrznego aktora}
Mimo tego, że Akka.TestKit udostępnia metody pozwalające na sprawdzenie stanu wewnętrznego aktora, nie jest to zalecane podejście. 
Prowadzi do sytuacji, gdzie testy zależą od wewnętrznej implementacji. 
Tak jak w tradycyjnych testach jednostkowych testuje się implementacje interfejsów, tak w Akka.net testuje się zachowania na poszczególne obsługiwane wiadomości.

\subsection{Używanie metody \lstinline{Ask} }
\lstinline{Ask<T>} jest metodą pozwalającą oczekiwać na wiadomość zwrotną z systemu aktorów danego typu \lstinline{T}.
Jest ona częścią wewnętrznych mechanizmów Akka.net i sami twórcy nie zalecają jej używania wewnątrz pisanych aktorów \cite{bib:AkkaNoAsk}. 
\lstinline{Ask<T>} nie pozwala na zmianę wątku, co w przypadku testów doprowadza do sytuacji, gdzie test zwraca błąd związany z przekroczeniem czasu oczekiwania na wiadomość zwrotną. Zamiast tego, w testach powinna występować metoda \lstinline{ExpectMsg<T>}, która zwraca pierwszą wiadomość typu \lstinline{T}, która przyjdzie do aktora testowego dostępnego w klasie \lstinline{TestKit}.  

\section{Scenariusze testowe}
Poza testami jednostkowymi, powstały też testy integracyjne łączące w jednym teście większość stworzonych aktorów. Testy te odzwierciedlają realne sytuacje spotykane w firmach wynajmujących biura. 

\subsection{Nagłe dodanie spotkania}
\begin{lstlisting}
    7:30 - w sali do spotkań panuje temperatura 25st.C
    Zaplanowane jest spotkanie na 10:30 kończące się o 12:00 z wymaganą temperaturą 21st.C
    
    8:30 - dyrektor dodaje spotkanie spotkanie z zespołem na 09:00 kończące się o 10:00 z wymaganą temperaturą 18st.C

    9:00 - w sali powinna panować temperatura 18st.C +- 0.5st.C
    
    10:00 - w sali powinna panować temperatura 18st.C +- 0.5st.C
    
    10:30 - w sali powinna panować temperatura 21st.C +- 0.5st.C
    
    12:00 - wszystkie urządzenia w sali powinny być wyłączone, 
    jako że nie ma przewidzianych innych spotkań na ten dzień
\end{lstlisting}

\subsection{Anulowanie spotkania}
\begin{lstlisting}
    7:30 - w sali do spotkań panuje temperatura 25st.C
    Zaplanowane są dwa spotkania:
     -spotkanie z klientem na 9:00 kończące się o 10:00 z wymaganą temperaturą 18st.C    
     -spotkanie zespołu na 10:30 kończące się o 12:00 z wymaganą temperaturą 21st.C
    
    8:30 - do recepcji dzwoni klient i odwołuje spotkanie
    
    9:00 - wszystkie urządzenia w sali powinny być wyłączone

    10:30 - w sali powinna panować temperatura 21st.C +- 0.5st.C

    12:00 - wszystkie urządzenia w sali powinny być wyłączone, 
    jako że nie ma przewidzianych innych spotkań na ten dzień
\end{lstlisting}

\subsection{Przesunięcie spotkania}
\begin{lstlisting}
    7:30 - w sali do spotkań panuje temperatura 25st.C
    Zaplanowane jest spotkanie na 9:00 kończące się o 12:00 z wymaganą temperaturą 18st.C
        
    8:30 - Spotkanie zostaje przesunięte na 11:00 do 14:00

    9:00 - wszystkie urządzenia w sali powinny być wyłączone
    
    11:00 - w sali powinna panować temperatura 18st.C +- 0.5st.C
   
    14:00 - wszystkie urządzenia w sali powinny być wyłączone, 
    jako że nie ma przewidzianych innych spotkań na ten dzień
\end{lstlisting}

\subsection{Przedłużenie spotkania}
\begin{lstlisting}
    7:30 - w sali do spotkań panuje temperatura 25st.C
    Zaplanowane jest spotkanie na 9:00 kończące się o 11:00 z wymaganą temperaturą 18st.C
            
    9:00 - w sali powinna panować temperatura 18st.C +- 0.5st.C
    
    10:45 - Spotkanie zostaje przedłużone do 12:00
    
    11:30 - w sali powinna panować temperatura 18st.C +- 0.5st.C
   
    12:00 - wszystkie urządzenia w sali powinny być wyłączone, 
    jako że nie ma przewidzianych innych spotkań na ten dzień
\end{lstlisting}

\subsection{Otwarcie okna - zmiana temperatury niezgodna z modelem}
\begin{lstlisting}
    7:30 - w sali do spotkań panuje temperatura 20st.C
    Zaplanowane jest spotkanie na 9:00 kończące się o 11:00 z wymaganą temperaturą 23st.C

    8:30 - Otwarcie okna powoduje spadek temperatury do 18st.C 
    
    9:00 - w sali powinna panować temperatura 23st.C +- 0.5st.C

    11:00 - wszystkie urządzenia w sali powinny być wyłączone, 
    jako że nie ma przewidzianych innych spotkań na ten dzień
\end{lstlisting}
