\chapter{System aktorów}\label{ch:aktorzy}
\section{Architektura systemu aktorów}
Systemy aktorów w akka.net mają strukturę drzewiastą. 
Każdy aktor jest dzieckiem aktora głównego lub innego aktora. 

\begin{figure}[ht!]
    \centering
    \tikzset{
        edge from parent fork down,        
        level distance = 2cm,
        minimum height=8mm,
        every tree node/.style={draw, rectangle, rounded corners, align=center}}
    \begin{tikzpicture}[]
    \Tree 
        [.Budynek   
            [.{Firma A} 
                [.{Pokój a}   
                    [. {Sensor\\Temperatury} 
                        Termometr 
                    ]
                    [.{Kontroler\\Temperatury} 
                        {Klimatyzator 1} 
                        {Klimatyzator 2}
                    ]                    
                ]
            ]
            [.{Firma B} 
                {\ldots}
            ]
            {Źródło czasu}
            {Źródło modeli}
        ]        
    \end{tikzpicture}
    \caption{Schemat systemu aktorów}
\end{figure}
 

Aktor główny symbolizuje modelowany budynek.
Jego dziećmi są aktorzy firm, którzy odpowiadają za odświeżanie informacji z kalendarzy i przechowywanie ewentualnych domyślnych wartości parametrów pomieszczeń.

Każda firma ma pod sobą kilku aktorów odpowiedzialnych za przypisane im pomieszczenia.
Aby móc wpływać na parametry pomieszczenia, jego aktor ma dostęp do danych o pomieszczeniu od agentów sensorów oraz może wysłać sygnały kontrolne do urządzeń za pomocą aktorów kontrolerów.

Oba powyższe rodzaje aktorów łączą się z urządzeniami, odpowiednio sensorami i aktuatorami.

Dodatkowo oprócz opisanej struktury, w systemie powienien występować co najmniej jeden aktor będący źródłem czasu oraz jeden będący źródłem modeli parametrów pomieszczenia.

\section{Źródło czasu}
Źródło czasu to aktor wysyłający sygnały ze znacznikiem czasowym w określonych odstępach. Wykorzystuje do tego model subskypcyjny. Może występować jako pojedyńczy aktor w całym systemie lub w każdym fizycznym urządzeniu. 

\input{./diagrams/timeSourceSingle}

Pierwszy scenariusz użycia ma miejsce w aplikacji symulatora, gdzie nie trzeba brać pod uwagę opóźnień w przesyłaniu wiadomości ze znacznikami czasowymi. \seefigure{timeSourceSingle}

\begin{figure}[ht!]
    \centering
    \tikzset{
        edge from parent fork down,        
        level distance = 2cm,
        minimum height=8mm,
        every tree node/.style={draw, rectangle, rounded corners, align=center}}
    \begin{tikzpicture}[]
    \Tree 
        [.Budynek   
            [.{Firma A} 
                [.{Pokój a}   
                    [. {Sensor\\temperatury} 
                        Termometr 
                    ]
                    [.{Kontroler\\temperatury} 
                        {Klimatyzator} 
                    ] 
                    {Źródło\\czasu}                   
                ]
            ]
            [.{Firma B} 
                [.{Pokój a}   
                    [. {Sensor\\temperatury} 
                        Termometr 
                    ]
                    [.{Kontroler\\temperatury} 
                        {Klimatyzator} 
                    ] 
                    {Źródło\\czasu}                   
                ]
            ]
            {Źródło\\modeli}
        ]        
    \end{tikzpicture}
    \caption{Schemat systemu aktorów dla wielu źródeł czasu}
    \label{fig:timeSourceMulti}    
\end{figure}
 

Drugi sposób zastosowania ogranicza ilość sygnałów przesyłanych pomiędzy fizycznymi urządzeniami co pozwala na lepszą skalowalność i daje aktorom odpowiedzalnym za pomieszczenia większą niezależność w wypadku awarii połączeń. \seefigure{timeSourceMulti}
Jest to zalecany sposób implementacji w rzeczywistym zastosowaniu.

\section{Źródło modeli}
Źródło modeli jest aktorem podobnym do źródła czasu. 
Jego zadaniem jest przyjmowanie zaktualizowanych danych o modelu parametrów pomieszczenia i rozesłanie ich do wszystkich aktorów korzystających z takiego modelu.
Może istnieć jeden na cały system lub w wielu kopiach na kilku poziomach w zależności od potrzeb dostosowywania modeli pod konkretne pomieszczenia.

\input{./diagrams/modelSourcePerSystem}

Gdy zależy nam na prostej aktualizacji modelu w całym systemie najlepiej stworzyć jedne źródło modeli dla całego systemu. \seefigure{modelSourcePerSystem}

\begin{figure}[ht!]
    \centering
    \tikzset{
        edge from parent fork down,        
        level distance = 2cm,
        minimum height=8mm,
        every tree node/.style={draw, rectangle, rounded corners, align=center}}
    \begin{tikzpicture}[]
    \Tree 
        [.Budynek   
            [.{Firma A} 
                [.{Pokój a}   
                    [. {Sensor\\Temperatury} 
                        Termometr 
                    ]
                    [.{Kontroler\\Temperatury} 
                        {Klimatyzator} 
                    ] 
                ]
                {Źródło\\modeli}                   
            ]
            [.{Firma B} 
                [.{Pokój a}   
                    [. {Sensor\\Temperatury} 
                        Termometr 
                    ]
                    [.{Kontroler\\Temperatury} 
                        {Klimatyzator} 
                    ] 
                ]
                {Źródło\\modeli}                   
            ]
            {Źródło\\czasu}
        ]        
    \end{tikzpicture}
    \caption{Schemat systemu aktorów dla źródła modeli per firma}
    \label{fig:modelSourcePerCompany}    
\end{figure}
 

Jeżeli poszczególne firmy mają inne rodzaje pomieszczeń, można przygotować oddzielne źródło modeli dla każdej z nich. \seefigure{modelSourcePerCompany}

\input{./diagrams/modelSourcePerRoom}

W wypadku, gdy mamy do czynienia z nietypowym pomieszczeniem najlepiej jest wydzielić mu oddzielne źródło modeli, aby móc dostosować modele bardziej przystające do tego pomieszczenia. \seefigure{modelSourcePerRoom}

\input{./diagrams/modelTraining}

Taka metoda wdrożenia źródeł modeli zastosowana dla każdego pomieszczenia oddzielnie pozwalałaby na zastosowanie uczenia maszynowego i pozwalałaby na powolne przystosowanie parametrów modelu dla każdego z pomieszczeń. 
% \seefigure{modelTraining}

\input{./diagrams/modelTraining}

\section{Firma}
Aktor firmy służy za dyspozytora wiadomości. Przekierowuje wymagania otrzymane od adaptera kalendarza firmowego do odpowiednich aktorów pokoi.
Jeżeli jakieś wymaganie nie posiada jakiegoś parametru, może nadać mu wartość domyślną dla całej firmy.

W przyszłości może też zbierać infromacje ze statusów poszczególnych pokoi aby wygenerować raport zużycia prądu.

\section{Pomieszczenie}
Aktor pokoju jest najważniejszym aktorem w całym systemie. 
Odpowiada za zbieranie informacji z różnych sensorów w pomieszczeniu i złożeniu ich w status pokoju, który jest wysyłany wszystkich zainteresowanych agentów.

To on jest ostatnim poziomem na którym aktor jest w stanie pracować niezależnie od pozostałych aktorów-rodzeństwa, a nawet bez łączności z rodzicem przez dłuższy czas. 

Dla zapewnienia największej niezależności najlepiej, aby aktorzy pokoi i wszyskie jego dzieci znajdowały się na tym samym (mikro)komputerze. Jak długo takie urządzenie będzie w stanie połączyć się podlegającymi mu sensorami oraz urządzeniami HVAC (np. za pomocą Bluetooth) utrata łączności z systemem głównym nie zaburzy jego pracy zgodznie z jego aktualnym stanem.

\section{Symulator pomieszczenia}

Symulator pomieszczenia jest aktorem dziedziczącym po aktorze pomieszczenia wszystkie zachowania, a dodatkowo pozwala na sterowanie parametrami pomieszczenia. Jest to część systemu stworznona tylko na potrzeby symulatora i w realnym zastosowaniu nie musi być użyta.

\section{Jednostka} 
Jednostki to kategoria aktorów którzy dzielą sporą ilość zachowań, głównie w celu komunikacji z urządzeniami - sensorami lub aktuatorami. Jedna jednostka powinna odpowiadać tylko za jeden parametr pomieszczenia. 

Możliwości jednostek:
\begin{itemize}
    \item Oczekiwania na dane początkowe.
    \item Wykonywanie akcji na sygnał czasowy lub co określoną ilość czasu.
    \item Wykorzystywanie modeli matematycznych oraz statusu pomieszczenia do wyliczania zmian wartości parametru.
\end{itemize}
Dokładny opis działania jednostek znajduje się w dodatku \ref{ch:unit}

\section{Sensor}
Sensor to aktor, który podaje wartość jednego z parametrów pomieszczenia. Aktora połączonego z rzeczywistym urządzeniem możemy otrzymać za pomocą jednostki cyklicznej \seeappendix{subsec:cyclicUnit}, która co jakiś czas odczytywałaby wartości parametru z urządzenia (np. termometru) zapisywałaby ją wewnątrz obiektu i rozesłała wszystkim subskrybentom. 

\section{Symulator sensora}
Symulator sensora różni się od zwykłego sensora tym, że nie ma urządzenia, z którego mógłby odczytać wartość parametru. 
Zamiast tego, aktor musi wyliczyć nową wartość parametru za pomocą matematycznego modelu. Do napisania takiego aktora należy użyć jednostki symulującej. \seeappendix{subsec:simUnit} 

Co pewnien odstęp czasu symulator używa modelu, aby policzyć zmianę wartości np. temperatury korzystając z ostatniego otrzymanego statusu pomieszczenia oraz informacji ile czasu minęło od znacznika czasowego podanego w statusie.

\section{Kontroler}
Kontroler jest aktorem przeciwstawnym do sensora. Decyduje on, jakie sygnały wysłać urządzeniom-aktuatorom, żeby w pomieszczeniu dany parametr osiągnął żądaną wartość.
Aby to wyliczyć on potrzebuje modelu parametru, zatem będzie on dziedziczył po jednostce symulującej. \seeappendix{subsec:simUnit}

\section{Urządzenia}
Urządenia przechowują informację o parametrach urządzenia tzw. definicję urządzenia oraz jego aktualny stan czyli który tryb jest aktywny i jaka docelowa temperatura jest ustawiona. 