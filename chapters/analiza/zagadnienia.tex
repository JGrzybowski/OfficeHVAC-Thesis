\section{Omówienie zagadnienia}
W większości budynków biurowych zamontowane są klimatyzatory, które zapewniają komfortowe warunki pracy osób przebywających w biurach. 
Ilość prądu potrzebnego do funkcjonowania sieci takich urządzeń jest sporą częścią miesięcznych kosztów dla właściciela budynku.

Przy ręcznym ustawianiu komfortowej temperatury przez człowieka zwykle odbywa się to na początku spotkania gdy uczestnicy uznają, że warunki w pomieszczeniu nie odpowiadają ich wymaganiom. Aby jak najszybciej pozbyć się tego uczucia włączają najmocniejszy, lecz nie koniecznie najbardziej oszczędny, tryb w klimatyzatorach.

Założeniem implementowanego w tej pracy systemu jest ustalanie temperatury komfortu przed spotkaniem i uruchomienie klimatyzatorów przed spotkaniem w najbardziej ekonomicznym wariancie, tak aby uczestnicy mieli komfortowe warunki od początku spotkania i utrzymanie ich przez całe spotkanie. 

Głównym źródłem informacji w których pomieszczeniach i w jakich godzinach odbywają się spotkania są kalendarze firmowe. Można wyobrazić sobie inne źródła takie jak lokalizacja pracowników przypisanych do pomieszczenia, jednak implementowany system będzie skupiał się głównie na wydarzeniach firmowych, jako że są mniej dynamiczne i można za ich pomocą zamodelować również np. spóźniającego się pracownika przesuwając godzinę spotkania.

Przewidzieć ile czasu wcześniej należy uruchomić urządzenia i w jakim trybie można za pomocą matematycznych modeli. Przygotowanie dokładniejszych modeli nie mieści się w zakresie pracy, przyjęto zatem prosty model opierający się o moc i wydajność urządzeń. 
