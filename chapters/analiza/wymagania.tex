\section{Analiza wymagań systemu}
Przy założeniu ewentualnej późniejszej rozbudowy systemu lub przeniesienia go na inny język programowania system musiał spełniać następujące wymagania:

\subsection*{Możliwość dodania nowych parametrów pomieszczenia}
Dodanie nowych parametrów takich jak wilgotność czy nasłonecznienie pozwalałoby udoskonalić model temperatur i zbliżyć go do rzeczywistego modelu fizycznego.

Jednocześnie dodanie takich parametrów jak poziom tlenu w pomieszczeniu dawałby możliwość sprawdzenia czy osobom w pomieszczeniu nie jest zbyt duszno oraz przewietrzenia pomieszczenia za pomocą dostępnych urządzeń.

Oczywiście każde dodanie takiego parametru wiązałoby się z nowym typem sensora i w niektórych przypadkach aktuatora o ile badane zajwisko zmianiałoby się na tyle wolno, że moglibyśmy na nie oddziałowywać.

\subsubsection*{Parametryzacja i podmiana modeli}
Możliwość parametryzacji model pozwoliłaby dostrajać modele i poprawić precyzję działania i sugesti systemu.

Podmiana modeli umożliwiłaby aktualizację parametrów modelu lub podmianę na inaczej skonstruowany model (np. przewidujący temperatury z innych parametrów pomieszczenia) bez wyłączania systemu . 

\subsection*{Możliwość podpięcia różnych źródeł danych}
W różnych firmach używa się różnych systemów do ustalania terminów spotkań czy rezerwacji sal np. Microsoft Exchange Server czy IBM Domino. 

System powinien pozwalać na dopisanie adaptera zajmującego się przekazywaniem wydarzeń do systemu HVAC i uniezależnianiem go od źródła danych.

\subsection*{Możliwość zbadania stanu systemu}
Dla celach diagnostycznych oraz aby dostroić modele potrzebne są dane o stanie urządzeń oraz dane z samego systemu HVAC. Musi zatem istnieć sposób, aby w łatwy sposób połączyć się z systemem HVAC i zerbać informacje o jego stanie.

\section{Analiza wymagań aplikacji symulatora}

\subsection{Schemat budynku}
\subsection{Scenariusze}
