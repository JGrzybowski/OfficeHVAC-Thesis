\chapter{Interakcje między aktorami}
Dla zwiększenia czytelności wszystkie diagramy w tym rozdziale traktują  interfejs użytkownika i model widoku jako całość. Są to komponenty spoza systemu aktorów łączące się z nim zawsze za pomocą aktorów-pomostów.

Dla opisu interakcji w systemie aktorów nie jest tak istotne jaki był czynnik inicjujący interakcje np. klik, w porównaniu do wiadomości która przyszła do systemu.

\section{Dodawanie/usuwanie pomieszczeń}
\subsection*{Dodanie pomieszczenia}
Interfejs użytkownika przekazuje do aktora firmy wszystkie potrzebne dane potrzebne do stworzenia aktora pokoju (przede wszystkim adres źródła czasu).
Pokój inicjalizuje się zapisując się na subskrypcje pod podanymi mu adresami.
Jednocześnie, Firma zapisuje się na subkrypcję Pokoju i po jego inicjalizacji będzie otrzymywać status pokoju. \seefigure{addRoom} 
\begin{figure}[ht!]
    \centering
    \begin{sequencediagram}
        \newinst[]{A}{Client}{}
        \newinst[]{B}{ServerA}{}
        \begin{call}{A}{Call()}{B}{}
        \end{call}
    \end{sequencediagram}
    \caption{Schamat dodawania pomieszczenia}
    \label{fig:addRoom}
\end{figure}
 

\subsection*{Usunięcie pomieszczenia}
Aby poprawnie usunąć pomieszczenie z systemu należy wysłać wiadmość o usunięciu pomieszczenia do aktora firmy, ten przesyła wiadomość do aktora pokoju i usuwa referencję z listy dostępnych mu pomieszczeń.
Aktor pokoju po otrzymaniu sygnału rozsyła sygnał zatrzymujący pracę do wszystkich swoich aktorów dzieci, a następnie sam zatrzymuje się. \seefigure{removeRoom}
\begin{figure}[ht!]
    \centering
    \begin{sequencediagram}
        \newinst[0]{UI}{\shortstack{Interfejs\\użytkownika}}{}
        \newinst[2]{Company}{Firma}{}
        \newinst[2]{Room}{Pokój}{}
        \newinst[1]{Child1}{Sensor}{}
        \newinst[1]{Child2}{Kontroler}{}
        \newinst[1]{Time}{\shortstack{Źródło\\czasu}}{}

        \begin{mess}{UI}{RemoveRoom}{Company} \end{mess}
        \begin{mess}{Company}{RemoveRoom}{Room} \end{mess}
            
        \begin{mess}{Room}{Unsubscribe}{Time} \end{mess}
        \begin{call}{Company}{}{Company}{\shortstack{Usuń pokój\\z listy}}\end{call}
            
        \begin{sdblock}{Wygaszanie aktora}{}
            \begin{mess}{Room}{Stop}{Child2} \end{mess}
            \begin{mess}{Room}{Stop}{Child1} \end{mess}
            \begin{call}{Room}{}{Room}{Stop} \end{call}
        \end{sdblock}
    \end{sequencediagram}
    \caption{Schemat usuwania pomieszczenia}
    \label{fig:removeRoom}
\end{figure}
 

\section{Dodawanie/usuwanie sensorów i kontrolerów}
Dodawanie oraz usuwanie sensorów i kontrolerów działa na podobnych mechanizmach dla obu rodzajów aktorów ze względu na wspólną klasę bazową.

\subsection*{Dodanie sensora lub kontrolera}
Przy dodawaniu jednostek symulacyjnych należy podać adresy aktorów, który zapewnią dane do inicjalizacji aktorów. (Rysunki \ref{fig:addTemperatureSensor} i \ref{fig:addTemperatureController}) Więcej szczegółów na temat inicjalizacji jednostek można znaleźć w sekcji \ref{subsec:unitInitialization}

\begin{figure}[ht!]
    \centering
    \begin{sequencediagram}
        \newinst[]{Room}{Room}{}
        \newinst[2]{Sensor}{TemperatureSensor}{}
        \newinst[2]{Model}{ModelParams}{}
        \newinst[]{Time}{TimeSimulator}{}

        \begin{mess}{Room}{Adresy subskrypcji}{Sensor}\end{mess}

        \begin{sdblock}{Subskrypcje sensora}{}
            \begin{mess}{Sensor}{Subscibe}{Room}\end{mess}
            \begin{mess}{Sensor}{Subscibe}{Model}\end{mess}
            \begin{mess}{Sensor}{Subscibe}{Time}\end{mess}
        \end{sdblock}

        \begin{sdblock}{Odpowiedzi na subskrupcje}{}
            \begin{mess}{Time}{TimeChanged}{Sensor}\end{mess}   
            \begin{mess}{Model}{TemperatureModel}{Sensor}\end{mess}
            \begin{mess}{Room}{RoomStatus}{Sensor}\end{mess}
        \end{sdblock}
                
        \begin{sdblock}{Podpięcie sensora}{}
            \begin{mess}{Sensor}{SensorAvaliable}{Room}\end{mess}
            \begin{mess}{Room}{Subscribe}{Sensor}\end{mess}
            \begin{mess}{Sensor}{TemperatureValue}{Room}\end{mess}
        \end{sdblock}
    \end{sequencediagram}
    \caption{Schemat dodawania symulatora sensora do pomieszczenia}
    \label{fig:addTemperatureSensor}
\end{figure}
 
\begin{figure}[ht!]
    \centering
    \begin{sequencediagram}
        \newinst[0]{Room}{Pokój}{}
        \newinst[2]{Controller}{\shortstack{Kontroler\\temperatury}}{}
        \newinst[2]{Model}{\shortstack{Źródło\\modeli}}{}
        \newinst[]{Time}{\shortstack{Źródło\\czasu}}{}

        \begin{mess}{Room}{Adresy subskrypcji}{Controller}\end{mess}

        \begin{sdblock}{Subskrypcje kontrolera}{}
            \begin{mess}{Controller}{Subscibe}{Room}\end{mess}
            \begin{mess}{Controller}{Subscibe}{Model}\end{mess}
            \begin{mess}{Controller}{Subscibe}{Time}\end{mess}
        \end{sdblock}

        \begin{sdblock}{Odpowiedzi na subskrupcje}{}
            \begin{mess}{Time}{TimeChanged}{Controller}\end{mess}   
            \begin{mess}{Model}{TemperatureModel}{Controller}\end{mess}
            \begin{mess}{Room}{RoomStatus}{Controller}\end{mess}
        \end{sdblock}
                
        \begin{sdblock}{Podpięcie kontrolera}{}
            \begin{mess}{Controller}{SensorAvaliable}{Room}\end{mess}
            \begin{mess}{Room}{Subscribe}{Controller}\end{mess}
            \begin{mess}{Controller}{TemperatureValue}{Room}\end{mess}
        \end{sdblock}
    \end{sequencediagram}
    \caption{Schemat dodawania kontrolera temperatury do pomieszczenia}
    \label{fig:addTemperatureController}
\end{figure}
 

\subsection*{Usunięcie sensora lub kontrolera}
Procedura usuwania sensora lub kontrolera przypomina bardzo schemat z usuwania pomieszczenia. Sygnał jest przesyłany do rodzica usuwanego aktora i przekierowywany do usuwanego aktora. Usuwany aktor wypisuje się z subskrypcji, na które się zapisał i zatrzymuje wszystkich podległych mu aktorów oraz siebie. (Rysunki \ref{fig:removeTemperatureSensor} i \ref{fig:removeTemperatureController})
\begin{figure}[ht!]
    \centering
    \begin{sequencediagram}
        \newinst[]{UI}{\shortstack{Interfejs\\użytkownika}}{}
        \newinst[]{Room}{Room}{}
        \newinst[2]{Sensor}{TemperatureSensor}{}
        \newinst[2]{Model}{ModelParams}{}
        \newinst[]{Time}{TimeSimulator}{}


        \begin{mess}{UI}{RemoveSensor}{Room}\end{mess}
        \begin{mess}{Room}{RemoveSensor}{Sensor}\end{mess}

        \begin{call}{Room}{}{Room}{Usuń sensor z listy}\end{call}            

        \begin{sdblock}{Subskrypcje sensora}{}
            \begin{mess}{Sensor}{Unsubscibe}{Room}\end{mess}
            \begin{mess}{Sensor}{Unsubscibe}{Model}\end{mess}
            \begin{mess}{Sensor}{Unsubscibe}{Time}\end{mess}
        \end{sdblock}

    \begin{call}{Sensor}{}{Sensor}{Stop}\end{call}
    \end{sequencediagram}
    \caption{Schemat usuwania symulatora sensora z pomieszczenia}
    \label{fig:removeTemperatureSensor}
\end{figure}
 
\begin{figure}[ht!]
    \centering
    \begin{sequencediagram}
        \newinst[]{UI}{\shortstack{Interfejs\\użytkownika}}{}
        \newinst[]{Room}{Room}{}
        \newinst[2]{Controller}{TemperatureController}{}
        \newinst[2]{Model}{ModelParams}{}
        \newinst[]{Time}{TimeSimulator}{}


        \begin{mess}{UI}{RemoveSensor}{Room}\end{mess}
        \begin{mess}{Room}{RemoveSensor}{Controller}\end{mess}

        \begin{call}{Room}{}{Room}{Usuń kontroler z listy}\end{call}            

        \begin{sdblock}{Subskrypcje kontrolera}{}
            \begin{mess}{Controller}{Unsubscibe}{Room}\end{mess}
            \begin{mess}{Controller}{Unsubscibe}{Model}\end{mess}
            \begin{mess}{Controller}{Unsubscibe}{Time}\end{mess}
        \end{sdblock}

    \begin{call}{Controller}{}{Controller}{Stop}\end{call}
    \end{sequencediagram}
    \caption{Schemat usuwania kontrolera temperatury z pomieszczenia}
    \label{fig:removeTemperatureController}
\end{figure}
 

\section{Dodawanie/usuwanie urządzeń}
Dodawanie i usuwanie urządzeń znacznie bardziej wpływa na działanie systemu, ponieważ może zmienić plan uruchamiania urządzeń. 

\subsection*{Dodanie urządzenia}
Pokój otrzymuje definicję urządzenia, którą przekazuje do odpowiedniego kontrolera. Ten tworzy dodaje urządzenie do listy i zleca planiście przeliczenie na planu prac. Dzięki temu może się okazać, że dzięki dodatkowemu urządzeniu dostosowywanie powietrza w pokoju będzie pochłaniać mniej energii elektrycznej. \seefigure{addDevice}
\begin{figure}[!htbp]
    \centering
    \begin{sequencediagram}
        \newinst[]{UI}{\shortstack{Interfejs\\użytkownika}}{}
        \newinst[]{Room}{Pokój}{}
        \newinst[]{Controller}{\shortstack{Kontroler\\temperatury}}{}
        \newinst[]{JobScheduler}{Planista}{}

        \begin{mess}{UI}{DeviceDefinition}{Room}\end{mess}
        \begin{mess}{Room}{DeviceDefinition}{Controller}\end{mess}

        \begin{call}{Controller}{}{Controller}{\shortstack{Dodaj urządzenia do listy}}\end{call}

        \begin{mess}{Controller}{TemperatureTask}{JobScheduler}\end{mess}
            
        \begin{mess}{JobScheduler}{Execution Plan}{Controller}\end{mess}              
        \begin{sdblock}{Jeżeli plan zakłada natychmiastową reakcję}{}
            \begin{call}{Controller}{}{Controller}{\shortstack{Przelacz urządzenia na\\odpowiedni tryb}}\end{call}
            \begin{mess}{Controller}{DevicesStatus}{Room}\end{mess}
        \end{sdblock} 
    \end{sequencediagram}
    \caption{Schemat dodawania urządzenia HVAC do pomieszczenia}
    \label{fig:addDevice}
\end{figure}
 

\subsection*{Usunięcie urządzenia}
Usunięcie urządzenia może spowodować, że pozostałe urządzenia w pomieszczeniu będą musiały pracować w trybach dostarczających więcej energii, ale jednocześnie mniej ekonomicznych. 
Po otrzymaniu sygnału o usunięciu urządzenia, aktor pokoju rozsyła wiadomość do wszystkich kontrolerów. Kontroler sterujący usuwanym urządzeniem wyłącza je, usuwa z listy i zleca przeliczenie planu pracy. \seefigure{removeDevice}
\begin{figure}[ht!]
    \centering
    \begin{sequencediagram}
        \newinst[]{UI}{\shortstack{Interfejs\\użytkownika}}{}
        \newinst[]{Room}{Pokój}{}
        \newinst[2]{Controller}{\shortstack{Kontroler\\temperatury}}
        \newinst[2]{JobScheduler}{Planista}{}

        \begin{mess}{UI}{RemoveDevice}{Room}\end{mess}
        \begin{mess}{Room}{RemoveDevice}{Controller}\end{mess}
        
        \begin{mess}{Controller}{TemperatureTask}{JobScheduler}\end{mess}      
        \begin{mess}{JobScheduler}{Execution Plan}{Controller}\end{mess} 
        
        \begin{sdblock}{Jeżeli plan zakłada natychmiastową reakcję}{}
            \begin{call}{Controller}{}{Controller}{\shortstack{Przelacz urządzenia na\\odpowiedni tryb}}\end{call}
            \begin{mess}{Controller}{DevicesStatus}{Room}\end{mess}
        \end{sdblock}
    \end{sequencediagram}
    \caption{Schemat usuwania urządzenia HVAC z pomieszczenia}
    \label{fig:removeDevice}
\end{figure}
 

\section{Upływ czasu}
Wiadomość o upływie czasu jest najczęściej przesyłanym sygnałem w systemie. Każdy aktor obsługuje ją inaczej. 

\subsection{Sensor}
Sensor sprawdza czy upłynęło wystarczająco dużo czasu, by wysłać nową wartość obsługiwanego parametru. \seefigure{timeChangedSensor}
\begin{figure}[ht!]
    \centering
    \begin{sequencediagram}
        \newinst[0]{Time}{\shortstack{Źródło\\czasu}}{}
        \newinst[]{Sensor}{Sensor}{}
        \newinst[6]{Subs}{Subskrybenci}{}

        \begin{mess}{Time}{TimeChanged}{Sensor}\end{mess}
        \begin{call}{Sensor}{\shortstack{Sprawdz ile czasu upłynelo}}{Sensor}{jesli $\Delta_t$ > okres aktualizacji}\end{call}
        \begin{mess}{Sensor}{ParameterValue}{Subs}\end{mess}
    \end{sequencediagram}
    \caption{Reakcja sensora na upływ czasu}
    \label{fig:timeChangedSensor}    
\end{figure}
 

\subsection{Kontroler}
Kontroler sprawdza znacznik czasowy z wiadomości i porównuje z jego aktualnym planem prac. Jeżeli jakaś praca na czas startu przed znacznikiem, a nie została jeszcze wystartowana, zostaje wdrożona na dostępnych urządzeniach. \seefigure{timeChangedController}
\begin{figure}[ht!]
    \centering
    \begin{sequencediagram}
        \newinst[0]{Time}{\shortstack{Źródło\\czasu}}{}
        \newinst[]{Controller}{Kontroler}{}
        \newinst[6]{Subs}{Subskrybenci}{}

        \begin{mess}{Time}{TimeChanged}{Controller}\end{mess}
        \begin{call}{Controller}{\shortstack{Sprawdz czy nie nalezy rozpoczac pracy}}{Controller}{jezeli tak, to uruchom urzadzenia}\end{call}
        \begin{mess}{Controller}{ControllerStatus}{Subs}\end{mess}
    \end{sequencediagram}
    \caption{Reakcja kontrolera na upływ czasu}
    \label{fig:timeChangedController}    
\end{figure}
 

\subsection{Scenarzysta}
Scenarzysta posiada listę wiadomości, które musi wysłać o wyznaczonym czasie do wyznaczonych aktorów. Po otrzymaniu wiadomości o upływie czasu sprawdza, które wiadomości nie zostały jeszcze wysłane, a ich moment wysyłki już minął lub jest równy znacznikowi czasowemu. Wiadomości te są rozsyłane w kolejności od najwcześniejszej do najpóźniejszej. \seefigure{timeChangedScreenwriter}
\begin{figure}[ht!]
    \centering
    \begin{sequencediagram}
        \newinst[0]{Time}{\shortstack{Źródło\\czasu}}{}
        \newinst[]{Screenwriter}{Scenarzysta}{}
        \newinst[6]{Target}{\shortstack{Adresaci\\wiadomości}}{}

        \begin{mess}{Time}{TimeChanged}{Screenwriter}\end{mess}
        \begin{call}{Screenwriter}{\shortstack{Znajdz wiadomości, ktore\\nalezy wyslac w obecnej chwili}}{Screenwriter}{}\end{call}
        \begin{mess}{Screenwriter}{Wiadomość}{Target}\end{mess}
    \end{sequencediagram}
    \caption{Reakcja scenarzysty na upływ czasu}
    \label{fig:timeChangedScreenwriter}    
\end{figure}
 

\section{Zmiana wartości parametów modelu}
Jeżeli źródło modeli dostanie wiadomość zmieniającą wartości parametrów modelu, buduje nowy obiekt modelu i udostępnia go wszystkim subskrybentom.
Może to wywołać przeliczenie planu prac w kontrolerach. Nie wpływa to na wartość aktualnej temperatury w sensorze, jako że ten aktualizuje wartość swojego parametru w odstępach czasowych i użyje nowego modelu przy kolejnej aktualizacji.

\section{Zmiany stanu wydarzeń w kalendarzu}
Gdy kontroler otrzyma listę wymagań przesyła ją w postaci zadania do podległego mu aktora-planisty, którego jedynym zadanim jest wykonanie obliczeń. Dzięki przekazaniu zadania podległemu aktorowi jednostka nie jest zablokowana podczas obliczeń i może wykonywać inne zadania. \seefigure{scheduling}
\begin{figure}[ht!]
    \centering
    \begin{sequencediagram}
        \newinst[]{Calendar}{\shortstack{Adapter do\\kalendarza}}{}
        \newinst[]{Company}{Firma}{}
        \newinst[]{Room}{Pokój}{}
        \newinst[]{Controller}{Kontroler}{}
        \newinst[]{JobScheduler}{Planista}{}

        \begin{mess}{Calendar}{CalendarEvents}{Company}\end{mess}
        
        \begin{mess}{Company}{Expectations}{Room}\end{mess}
        \begin{mess}{Room}{Expectations}{Controller}\end{mess}
        
        \begin{mess}{Controller}{Tasks}{JobScheduler}\end{mess}
        \begin{mess}{JobScheduler}{ExecutionPlan}{Controller}\end{mess}

        \begin{sdblock}{Jeżeli zmiany wymagają natychmiastowej reakcji}{}
            \begin{call}{Controller}{}{Controller} {\shortstack{Przestaw\\urządzenia}}\end{call}
            \begin{mess}{Controller}{DevicesStatus}{Room}\end{mess}
            \begin{mess}{Room}{RoomStatus}{Company}\end{mess}
        \end{sdblock}
    \end{sequencediagram}
    \caption{Schemat dodawania subskrybenta}
    \label{fig:scheduling}    
\end{figure}
 

Czynnością wykonywaną w tym czasie mogłoby być np. okresowa weryfikacja czy zmiana wartości parametru postępuje w założonym tempie.
Po otrzymaniu od aktora-planisty definicji pracy jaką ma wykonać, kontroler ustawia odpowiednio urządzenia-aktuatory w sposób zależny od typu parametru.


