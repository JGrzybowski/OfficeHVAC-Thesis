\chapter{Wzorce projektowe wykorzystane podczas realizacji projektu}
\section{Rekwizyty}
Rekwizyt (ang. props) to obiekt w akka.net pozwalający na przechowywanie sposobu tworzenia obiektu aktora. 
Są wykorzystywane przy zarówno tworzeniu jak i ponownym uruchamianiu aktorów po nastąpieniu nieobsłużonego wyjątku.
Tylko one mają bezpośredni dostęp do kontruktora aktora. W momencie tworzenia aktora wykorzystywany jest obiekt rekwizytów i wywoływane jest ukryte wewnątrz wyrażenie z konstruktorem.

\section{Model subskrypcyjny}
Model subskrypcyjny (ang. subscribe/unsubscribe) to sposób komunikacji między agentami. Składa się z właściciela subskrypcji oraz subskrybentów. 
\definition{Subskrypcja to wiadomość rozsyłana przez jej właściciela do wszystkich którzy się złosili.}
\definition{Subskrybentem nazywamy agenta, który zgłosił właścicielowi subskrypcji chęć otrzymywania subskrypcji.

\begin{figure}[ht!]
    \centering
    \begin{sequencediagram}
        \newinst[2]{Owner}{Właściciel subskrybcji}{}
        \newinst[2]{Nowy}{Nowy subskrybent}{}
        \newinst[2]{Stary}{Subskrybent}{}

        \begin{mess}{Nowy}{Subscibe}{Owner}\end{mess}
        \begin{mess}{Owner}{Newsletter}{Nowy}\end{mess}
    \end{sequencediagram}
    \caption{Schemat dodawania subskrybenta}
    \label{fig:subscribe}
\end{figure}
 
W realizowanym systemie właściciel subskrypcji wysyła ostatnią rozesłąną subskrypcję do agenta, który zapisał się na subskrypcję. Dzięki temu dopiero co zapisany agent nie musi czekać na kolejny syngał inicjalizujący rozesłanie subskrypcji.
 
\begin{figure}[ht!]
    \centering
    \begin{sequencediagram}
        \newinst[1]{TriggerSource}{Źródło sygnału}{}
        \newinst[2]{Owner}{Właściciel subskrybcji}{}
        \newinst[1]{Sub1}{Subskrybent A}{}
        \newinst[1]{Sub2}{Subskrybent B}{}

        \begin{mess}{TriggerSource}{Trigger}{Owner}\end{mess}
        \begin{mess}{Owner}{Newsletter}{Sub1}\end{mess}
        \begin{mess}{Owner}{Newsletter}{Sub2}\end{mess}
    \end{sequencediagram}
    \caption{Schemat rozsyłania subskrypcji}
    \label{fig:subscriptionTrigger}
\end{figure}
 
Właściciel subskrypcji może rozsyłać subskrypcję zarówno na sygnał czasowy jak i sygnał przychodzący z zewnątrz. 

\section{Udostępnianie informacji o stanie aktora}
Aby móc w łatwy sposób odczytywać stan wewnętrzny systemu, większość jego aktorów dziedziczy po specjalnie napisanym typie aktora ???DebugableActor???. Ta klasa pozwala na przygotowanie informacji, która będzie rozsyłana za pomocą specjalnej subskrypcji diagnostycznej. Zadanie wybrania momentów w których rozsyłana jest subskrypcja należy do osoby rozszerzającej tą klasę. 

Metody dostępne do rozszerzenia lub użycia w klasie ???DebuggableActor???
\begin{enumerate}
    \item GenerateInternalState() - tworzy obiekt przedstawiający stan wewnętrzny aktora (metoda do rozszerzenia). 
    \item InformAboutInternalState() - rozsyła subskrypcję stanu wewnętrznego do wszystkich diagnostycznych subskrybentów.
    \item InformDebugSubscribers(object x) - Pozwala rozesłać do subskrybentów diagnostyczych inną wiadomość niż tą generowaną za pomocą GenerateInternalState()
    \item SetInternalStatus() - Metoda pozwalająca na nadgranie statusu wewnętrznego aktora np. w celu testowania. Domyślnie nie zmienia stanu aktora a jedynie wywołuje metodę InformAboutInternalState(). (metoda do rozszerzenia)
\end{enumerate}

\section{Aktor-pomost} \label{sec:aktor-pomost}
Aktor-pomost (ang. BridgeActor) to aktor pośredniczący w komunikacji pomiędzy interfejsem użytkownika symulatora a właściwym agentem z właściwego systemu. 

\subsection*{Tworzenie agenta-pomostu}
Powinien mieć przypisanego tylko jednego agenta z systemu i tylko jeden model widoku z interfejsu użytkownika. Przy tworzeniu pomost zapisuje się na subskrypcję diagnostyczną, ab odbierać informacje o wszelkich możliwych zmianach stanu aktora, do którego jest pośrednikiem.
\begin{figure}[ht!]
    \centering
    \begin{sequencediagram}
        \newinst[]{ViewModel}{Model widoku}{}
        \newinst[3]{Bridge}{Aktor-pomost}{}
        \newinst[3]{Actor}{Aktor}{}

    \begin{mess}{ViewModel}{<Model widoku,Adres Aktora>}{Bridge}\end{mess}
    \begin{mess}{Bridge}{DebugSubscribe}{Actor}\end{mess}
    \begin{mess}{Actor}{InternalStatus}{Bridge}\end{mess}
    \end{sequencediagram}
    \caption{Schemat tworzenia aktora-pomostu}
\end{figure}
 

Głównym zadaniem aktora-pomostu jest przesyłanie wiadomości do aktorów wewnątrz systemu zarządzającego urządzeniami oraz aktualizacja danych widocznych w aplikacji.

\subsection*{Przekazywanie wiadomości}
\begin{figure}[ht!]
    \centering
    \begin{sequencediagram}
        \newinst[1]{UI}{Interfejs użytkownika}{}
        \newinst[1]{ViewModel}{Model widoku}{}
        \newinst[1]{Bridge}{Aktor-pomost}{}
        \newinst[1]{Actor}{Aktor}{}

    \begin{mess}{UI}{AddSensor Button Click}{ViewModel}\end{mess}
    \begin{mess}{ViewModel}{AddSensor}{Bridge}\end{mess}
    \begin{mess}{Bridge}{AddSensor}{Actor}\end{mess}
    \end{sequencediagram}
    \caption{Schemat przekazywania wiadomości przez aktora-pomost}
    \label{fig:bridgeActorForwarding}
\end{figure}
 
Po akcji użytkownika, model widoku przesyła wiadomość do aktora-pomostu, który przekazuje wiadomość dalej do przypisanego mu agenta. Czasami model widoku może oczekiwać odpowiedzi od systemu (tzw. Ask) jednak jest to widoczne tylko na poziomie widoku modelu i blokuje tylko wątek odpowiedzialny za obsługę danego kliknięcia/edycji.

\subsection*{Aktualizacja interfejsu użytkownika}
\begin{figure}[ht!]
    \centering
    \begin{sequencediagram}
        \newinst[1]{UI}{Interfejs użytkownika}{}
        \newinst[1]{ViewModel}{Model widoku}{}
        \newinst[1]{Bridge}{Aktor-pomost}{}
        \newinst[1]{Actor}{Aktor}{}

    \begin{mess}{Actor}{Room Status}{Bridge}\end{mess}
    \begin{mess}{Bridge}{UpdateFromStatus}{ViewModel}\end{mess}
    \begin{mess}{ViewModel}{PropertyChanged}{UI}\end{mess}
    \end{sequencediagram}
    \caption{Schemat aktualizowania interfejsu użytkownika przez aktora-pomost}
\end{figure}
 
Wiadomość od agenta z systemu przychodzi do agenta-pomostu. Ten, wywołuje przygotowaną w modelu widoku metodę do aktualizacji wartości wyświetlanych.
Po wykonaniu metody uruchamia się zdarzenie aktualizujące wyświetlane wartości w interfejsie użytkownika.

\section{Wiadomości typu Request/Update/Value}
Do wygodnej obsługi odpytywania systemu oraz możliwości podmiany parametrów w symulatorze powstał schemat nazewnictwa wiadomości.
\begin{itemize}
    \item Value - Wiadomość zawierająca wartość parametru.
    \item Request - Wezwanie do wysłania wartości danego parametru. Odbiorca powinien odesłać wiadomość typu Value.
    \item Update - Wiadmość zlecająca nadpisanie wartości danego parametru u odbiorcy na wartość podaną w wiadomości.
\end{itemize}


\section{Obiekty typu Wymaganie/Zadanie/Praca}
Obsługa oczekiwanych parametrów została podzielona na trzy etapy. Wymagania zostają rozbite na Zadania dla kontroleró poszczególnych parametrów, a następnie modele pozwalają wyliczyć najbardziej optymalną konfigurację urządzeń, zapisaną w obiekcie Pracy. 

\begin{itemize}
    \item Wymaganie - obiekt opisujący kiedy, który parametr i jaka jego wartość jest oczekiwana.
    
    \item Zadanie - obiekt opisujący ograniczenia w których musi być spełnione Wymaganie. 

    Np. TemperatureTask to ilość czasu, obecna temperatura w pomieszczeniu i temperature jaką chcemy osiągnąć w przeciągu ww. czasu.
    
    \item Praca - obiekt określający najlepszą znalezioną konfigurację dla zadanego Zadania. 

    Np. TemperatureJob składa się z informacji: w którym trybie, od kiedy do kiedy, i na jaką temperaturę nastawić klimatyzatory.
\end{itemize}
