\chapter{Wprowadzenie}
\section{Podstawowe pojęcia}

\definition{HVAC to skrót od Heating, Ventilation, Air Conditioning czyli ogrzewania, wentylacji i klimatyzacji. Jest to grupa urządzeń odpowidzialnych za dostosowywanie parametrów powietrza (najczęściej temperatury) w pomieszczeniu.}

\definition{Sensor to urządzenie pozwalające komputerowi badać stan świata rzeczywistego np. skaner terenu tworzący wirtualną reprezentację ukształtowania powierzchni księżyca.}

\definition{Aktuator to urządzenie pozwalające komputerowi wpływać na stan świata rzeczywistego np. ramię robota przenoszące obiekty. }

\section{Omówienie zagadnienia}
W większości budynków biurowych zamontowane są klimatyzatory, które zapewniają komfortowe warunki pracy osób przebywających w biurach. 
Ilość prądu potrzebnego do funkcjonowania sieci takich urządzeń jest sporą częścią miesięcznych kosztów dla właściciela budynku.

Przy ręcznym ustawianiu komfortowej temperatury przez człowieka zwykle odbywa się to na początku spotkania, gdy uczestnicy uznają, że warunki w pomieszczeniu nie odpowiadają ich wymaganiom. Aby jak najszybciej pozbyć się tego uczucia, włączają najmocniejszy, lecz nie koniecznie najbardziej oszczędny, tryb w klimatyzatorach.

Założeniem systemu implementowanego w tej pracy jest ustalanie temperatury komfortu przed spotkaniem i uruchomienie klimatyzatorów przed spotkaniem w najbardziej ekonomicznym wariancie tak, aby uczestnicy mieli komfortowe warunki od początku aż do końca spotkania. 

Głównym źródłem informacji w których pomieszczeniach i w jakich godzinach odbywają się spotkania są kalendarze firmowe. Można wyobrazić sobie inne źródła, jak np. lokalizacja pracowników przypisanych do pomieszczenia. Jednak implementowany system będzie skupiał się głównie na wydarzeniach firmowych, jako że są mniej dynamiczne i można za ich pomocą zamodelować również np. spóźniającego się pracownika przesuwając godzinę spotkania.

Matematyczne modele zjawisk, takich jak wymiana ciepła, pozwalają przewidzieć w jakim trybie i ile czasu wcześniej należy uruchomić urządzenia.
