Aby móc wykonywać powtarzalne próby wydajnie potrzebne były:

\section{Analiza wymagań aplikacji symulatora}
Użycie rzeczywistych urządzeń HVAC, w celu sprawdzenia poprawności działania systemu, nie było możliwe w ramach tej pracy ze względu na kosztowność i czasochłonność takiego rozwiązania. 
Potrzebny był zatem symulator który zawierałby w sobie system HVAC oraz interfejs użytkownika umożliwiający interakcję z systemem. 

\subsection*{Symulacja upływu czasu}
Symulator powinien też pozwalać na przyspieszenie czasu w symulowanym systemie, ograniczyć czas potrzebny na testowanie.
i nie czekać na wynik działania systemu po np. dwóch godzinach. System powinien pozwalać na przeskalowanie czasu upływającego w systemie np. jedna sekunda czasu rzeczywistego to 5 minut czasu symulatora.

\subsection*{Mechanizm ładowania schematu budynku}
Aplikacja symulatora powinna pozwalać na zapisanie i ponowne załadowanie właściwości pomieszczeń wraz ze stanem urządzeń zanjdujących się w nim.

\subsection*{Silnik scenariuszy}
Scenariusze czyli lista sygnałów, które odbiera system o określonym czasie. Np. podniesienie się temperatury czy przesunięcie spotkania. Za pomocą odtwarzalnych scenariuszy możemy wielokrotnie testować zachowanie systemu w tych samych warunkach, ale np. z innymi parametrami w modelu temperatury.