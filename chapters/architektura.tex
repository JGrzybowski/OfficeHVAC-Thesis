\chapter{Architektura rozwiązania}
\section{Struktura systemu zarządzania urządzeniami}
\subsection{Obraz ogólny i kontekst działania systemu}
DIAGRAM OGÓLNY SYSTEMU
% \input{./diagrams/systemConcept}

W najogólniejszym obrazie, system odpowiada za przetwarzanie danych zebranych z sensorów rozmieszczonych w pomieszczeniach oraz informacji zapisanych w kalendarzach firm (syngały te mogą dotyczyć takich rzeczy jak np. zmiana godziny rozpoczęcia spotkania) na sygnały sterujące wysyłane do urządzeń pozwalających zachować w pomieszczeniach odpowiednie parametry, takie jak temperatura czy poziom ilości tlenu w powietrzu.

\subsection{Moduły systemu}
DIAGRAM MODUŁÓW SYSTEMU
% \input{./diagrams/systemModules}

W budowie samego systemu zarządzania urządzeniami HVAC możemy wyróżnić kilka modułów.
\begin{enumerate}
    \item Modele - zestaw abstrakcji i klas bazowych do budowy pozostałych modułów.  
    \item Jednostki - zestaw klas do budowy aktorów odpowiedzialnych za komunikację lub symulację urządzeń.
    \item Odczyt i kontrola parametrów - moduł zawierający matematyczny model zmiany temperatury wraz z aktrorami odpowiedzialnynmi za wysyłanie klimatyzatorom odpowiednich dyspozycji.
    \item Symulacja zmiany parametrów - matematyczne modele pozwalające przewidzieć zmiany parametrów wywołane działaniem aktuatorów.
    \item Źródła wymagań - zestaw adapterów przetwarzających dane nadesłane do systemu na wymagania dla poszczególnych parametrów. 
    \item Zarządzanie Pomieszczeniem - moduł odpowiedzialny za zarządzanie komunikacją pomiędzy sensorami a aktuatorami. 
\end{enumerate}

Należy też wyróżnić moduł aplikacji hostującej system aktorów. 
Może być to zarówno aplikacja desktopowa jak i webowa. 

\section{Struktura aplikacji symulatora}
Aplikacja symulatora hostująca system agentowy została zrealizowana jako aplikacja desktopowa w technologii WPF - Windows Presentation Foundation za pomocą wzorca MVVM - Model - View - ViewModel.
\subsection{Wzorzec MVVM}
Wzorzec ten wyróżnia trzy wartwy aplikacji \seefigure{mvvm}:
\begin{enumerate}
    \item Widok (View) - układ interfejsu użytkownika wraz z kontrolkami. 
    \item Model Widoku (ViewModel) - obiekt odpowiedzialny za pobieranie danych z Modelu i podawanie ich odpowiednio przetworzonych do Widoku. 
    \item Model - to dowolne źródła danych, do których Widok Modelu musi mieć dostęp, aby otrzymać potrzebne mu informacje. 
\end{enumerate}
\begin{figure}[ht!]
    \centering
    \tikzstyle{applayer} = [draw, rectangle,minimum width= 5cm, minimum height= 2cm]    
    \begin{tikzpicture}[]
        \node[applayer] (V)  {View};
        \node[applayer, below = of V] (VM) {ViewModel};
        \node[applayer, below = of VM] (M)  {Model};

        \path [draw, ->] (V.south west) [bend right] to node[left]{\shortstack{Zmienia wartości wewnątrz\\modelu widoku}} (VM.north west);
        \path [draw, ->] (VM.north east) [bend right] to node[right] {\shortstack{Powiadamia widok\\o potrzebie odświeżenia}} (V.south east);

        \path [draw, ->] (VM.south west) [bend right] to node[left]{\shortstack{Wysyła zmiany w danych\\lub prosi o aktualne dane}} (M.north west);
        \path [draw, ->] (M.north east) [bend right] to node[right] {\shortstack{Odsyła dane}} (VM.south east);
    \end{tikzpicture}
    \caption{Warstwy aplikacji we wzorcu MVVM}
    \label{fig:mvvm}    
\end{figure}
 

Po wykonaniu akcji przez użytkownika np. kliknięcia przycisku na Widoku, odpowienio przypisana komenda z Modelu Widoku zostaje wywołana. 
Może ona wywoływać metody z Modelu np. wysłać zapytanie do bazy danych lub przesłać wiadomość do aktora w systemie aktorów.
Jeżeli taka metoda zwróci jakieś dane Model Widoku przetwarza otrzymane dane i zapisuje wewnątrz swojego obiektu. 
Następnie powiadamia Widok o tym, że wartość jednej lub więcej właściwości zmieniła się i należy odświeżyć ją na interfejsie użytkownika \seefigure{mvvmExample}.
\begin{figure}[ht!]
    \centering
    \begin{sequencediagram}
        \newinst[1]{View}{Widok}{}
        \newinst[2]{ViewModel}{Model widoku}{}
        \newinst[2]{Model}{Model}{}

        \begin{call}{View}{RefreshClicked}{ViewModel}{DataChanged}
            \begin{call}{ViewModel}{GetData}{Model}{Data}
                \begin{call}{Model}{}{Model}{\shortstack{Pobierz dane np. \\z bazy danych}}\end{call}
            \end{call}
            \begin{call}{ViewModel}{}{ViewModel}{\shortstack{Przechowaj dane w polu}}\end{call}      
        \end{call}
        \begin{call}{View}{}{View}{\shortstack{Ponownie wyrenderuj dane}}\end{call}      
    \end{sequencediagram}
    \caption{Przykład komunikacji wewnątrz aplikacji zgodnej z MVVM}
    \label{fig:mvvmExample}
\end{figure}
 

System aktorów w aplikacji symulatora należy do warstwy Modelu \seefigure{mvvmSimulator}.
W aplikacji symulatora za komunikację między obiektami Modeli Widoków a systemem aktorów odpowiedzialni są aktorzy-pomosty, opisani dokładniej w sekcji \ref{sec:aktor-pomost}

\begin{figure}[ht!]
    \centering
    \begin{sequencediagram}
        \newinst[1]{View}{Widok}{}
        \newinst[2]{ViewModel}{Model widoku}{}
        \newinst[0]{Bridge}{Aktor-pomost}{}
        \newinst[2]{Actor}{Aktor}{}

        \begin{messcall}{View}{RefreshClicked}{ViewModel}\end{messcall}
            \begin{mess}{ViewModel}{GetData}{Bridge}\end{mess}
            \begin{mess}{Bridge}{GetData}{Actor}\end{mess}
            \begin{mess}{Actor}{Data}{Bridge}{}\end{mess}
            \begin{mess}{Bridge}{Update data}{ViewModel}\end{mess} 
            \begin{call}{ViewModel}{}{ViewModel}{\shortstack{Przechowaj dane w polu}}\end{call}      
            \begin{messcall}{ViewModel}{DataChanged}{View}\end{messcall}
        \begin{call}{View}{}{View}{\shortstack{Ponownie wyrenderuj dane}}\end{call}      
    \end{sequencediagram}
    \caption{Schemat tworzenia aktora-pomostu}
    \label{fig:mvvmSimulator}
\end{figure}
 
