\chapter{Struktura systemu zarządzania urządzeniami}

\section{Obraz ogólny i kontekst działania systemu}

W najogólniejszym obrazie, system odpowiada za przetwarzanie danych zebranych z sensorów rozmieszczonych w pomieszczeniach oraz informacji zapisanych w kalendarzach firm (syngały te mogą dotyczyć takich rzeczy jak np. zmiana godziny rozpoczęcia spotkania) na sygnały sterujące wysyłane do urządzeń HVAC aby zachować w pomieszczeniach odpowiednie parametry, takie jak temperatura czy poziom ilości tlenu w powietrzu. \seefigure{generalContextDiagram}

\section{Moduły systemu}
W budowie samego systemu zarządzania urządzeniami HVAC możemy wyróżnić kilka modułów.
\begin{enumerate}
    \item Modele - zestaw abstrakcji modelujących realy świat, użyty do budowy pozostałych modułów.  
    \item Jednostki - zestaw klas do budowy aktorów odpowiedzialnych za komunikację z sensorami i aktuatorami.
    \item Odczyt i kontrola parametrów - moduł zawierający matematyczny model zmiany temperatury wraz z aktrorami odpowiedzialnynmi za wysyłanie klimatyzatorom odpowiednich dyspozycji.
    \item Symulacja zmiany parametrów - matematyczne modele pozwalające przewidzieć zmiany parametrów wywołane działaniem aktuatorów.
    \item Źródła wymagań - zestaw adapterów przetwarzających dane nadesłane do systemu na wymagania dla poszczególnych parametrów. 
    \item Zarządzanie Pomieszczeniem - moduł odpowiedzialny za zarządzanie komunikacją pomiędzy sensorami a aktuatorami. 
\end{enumerate}

Należy też wyróżnić moduł aplikacji hostującej system aktorów. 
Może być to zarówno aplikacja desktopowa jak i webowa. 

\section{Architektura systemu aktorów}
Systemy aktorów w Akka.net mają strukturę drzewiastą. 
Każdy aktor jest dzieckiem aktora głównego lub innego aktora. 

\begin{figure}[ht!]
    \centering
    \tikzset{
        edge from parent fork down,        
        level distance = 2cm,
        minimum height=8mm,
        every tree node/.style={draw, rectangle, rounded corners, align=center}}
    \begin{tikzpicture}[]
    \Tree 
        [.Budynek   
            [.{Firma A} 
                [.{Pokój a}   
                    [. {Sensor\\Temperatury} 
                        Termometr 
                    ]
                    [.{Kontroler\\Temperatury} 
                        {Klimatyzator 1} 
                        {Klimatyzator 2}
                    ]                    
                ]
            ]
            [.{Firma B} 
                {\ldots}
            ]
            {Źródło czasu}
            {Źródło modeli}
        ]        
    \end{tikzpicture}
    \caption{Schemat systemu aktorów}
\end{figure}
 

Aktor główny symbolizuje modelowany budynek.
Jego dziećmi są aktorzy firm, którzy odpowiadają za odświeżanie informacji z kalendarzy i przechowywanie ewentualnych domyślnych wartości parametrów pomieszczeń.

Każda firma ma pod sobą kilku aktorów odpowiedzialnych za przypisane im pomieszczenia.
Aby móc wpływać na parametry pomieszczenia, jego aktor ma dostęp do danych o pomieszczeniu od agentów sensorów oraz może wysłać sygnały kontrolne do urządzeń za pomocą aktorów kontrolerów.

Oba powyższe rodzaje aktorów łączą się z urządzeniami, odpowiednio sensorami i aktuatorami.

Dodatkowo oprócz opisanej struktury, w systemie powienien występować co najmniej jeden aktor będący źródłem czasu oraz jeden będący źródłem modeli parametrów pomieszczenia.
