\chapter{Systemy agentowe}
\section{Model aktorów}
Model aktorów to model równoległości oparty na koncepcji aktorów, którzy przesyłają między sobą wiadomości.
Model ten zakłada, że każdy z aktorów jest samodzielną jednostką posiadającą kolejkę wiadomości. Wiadomości sa przetwarzane po jednej na raz w kolejności, w której przyszły. W ramach przetwarzania mogą nastąpić zmiana stanu wewnętrznego aktora lub wysłanie jednej lub więcej wiadomości do jednego lub więcej aktorów. 

Model zakłada jak najmniejszą odpowiedzialność pojedyńczego aktora, aby umożliwić jak najwyższą równogległość i skalowalność.

\section{Przegląd rozwiązań agentowych dla środowska .NET}
\subsection{Boris.NET}
Boris jest biblioteką do tworzenia systemów agentowych stworzoną w takcie badania metod projektowania systemów agentowych w Teesside University, Wielka Brytania. Protokół Borisa pozwala na łączenie agentów napisanych za pomocą różnych języków takich jak C++, Lisp czy Java w jeden system. Boris.NET jest biblioteką napisaną przez Aliego Bojarpour do obsługi biblioteki Boris za pomocą języków \csh i \fsh. 

Elastyczność Borisa pod względem ilości obsługiwanych jest cenną cechą, gdyż pozwalałaby na pisanie adapterów i dodatkowych funkcjonalności przez różne zespoły. 
Niestety, ani Boris, ani Boris.NET nie są projektami open-source, co uniemożliwia analizę i poznanie mechanizmów wewnętrznych biblioteki. Podobnie rzecz ma się ze wsparciem społeczności, która ogranicza sie do osób ze środowiska akademickiego.

Brak też kompleksowej dokumentacji, która jest potrzebna przy poznawaniu nowej architektury oprogramowania, jaką jest podejście agentowe.

\subsection{Orleans}
Orleans to biblioteka pozwalająca na pisanie rozproszonych aplikacji stworzona przez Microsoft Research.
Powstała z myślą o osobach zaczynających pracę z systemami agentów. 
Orleans stworzyło własny model oparty na ziarnach (ang. grains) rozproszonych pomiędzy silosami. Każdy silos jest najczęściej oddzielnym serwerem.
  
\subsection{Akka.net}
Akka.net jest portem javowej biblioteki Akka, służącej do tworzenia systemów aktorów. 
Co kilka miesięcy wydawana jest nowa wersja biblioteki rozszerzona o nowe możliwości. 
Aktorzy są połączeni w drzewiastą strukturę, jednak to nie ogranicza ich możliwości wysyłania wiadomości. 

\section{Porównanie Akka.net i Orleans}
Tabela \ref{tab:AkkaVsOrleans} przedstawia porównianie bazujące na przeglądzie dr. Rolanda Kuhna \cite{bib:AkkaVsOrleans}, pozwalające ocenić, które rozwiązanie bardziej nadaje się do realizacji założeń projektu.

\subsection*{Kontrola nad systemem i lokalizacją aktorów}
Niektóre z możliwości Orleans mogą być interesujące dla osób zaczynających z programowaniem równoległym. 
Pisanie kodu, który ukrywa rozproszenie systemu jest dużym ułatwieniem. 
Wiele rzeczy dzieje się automatycznie lub nie mamy na nie wpływu. 
Pod tym względem Akka.net najbardziej różni się od Orleans. Wymaga się znajomości założeń, które przyjęli autorzy biblioteki. W zamian programiści mają pełną kontrolę nad tym, co dzieje się w systemie.

W projektowanym systemie ważne są między innymi lokalizacja aktorów (aby wysyłać sygnały do klimatyzatorów np. podczerwienią lub bluetooth) oraz możliwość ustrukturyzowania systemu. 
Takie możliwości daje Akka.net. Orleans jest dobry do zastosowań czysto obliczeniowych, jednak pewnych wymaganiach jego konfiguracja byłaby uciążliwa lub wręcz niemożliwa do zrealizowania.

\subsection*{Wsparcie społeczności}
Kolejnym argumentem za wykorzystaniem Akka.net jest jej cykl rozwoju.
Akka.net jest wspierana przez firmę Petabridge, której pracownicy dodają nowe funkcjonalności oraz konsultują ze społecznością rozwój Akka.net jak i oryginalnej Akki. 

Społeczność Orleans od września 2016 opracowuje nową wersję biblioteki \cite{bib:Orleans2}. Nie został podany zakres zmian w API, które zajdą wraz z tą aktualizacją. 

\subsection*{Kolejność dostarczania wiadomości}
W Akka.net kolejność dostarczenia wiadomości jest zachowywana dla każdej pary aktorów. Nie znaczy to co prawda, że wszystkie wiadomości będą przychodzić w kolejności w jakiej zostały wysłane, ale zapobiegnie to sytuacjom, gdzie wiadomość z aktualizacją stanu sprzed chwili mogłaby nadgrać najbardziej aktualny status.

Orleans nie posiada takiego mechanizmu. Jego założeniem jest, to że ziarna wykonują zlecone im obliczenia i kolejność ich wykonywania nie powinna zaburzać ich pracy.

\subsection*{Uzasadnienie wyboru Akka.net}
Większa kontrola nad systemem, stabilność API połączone z podobieństwem struktury systemu Akka.net do struktury budynku zadecydowały o wyborze Akka.net do implementacji systemu oraz symulatora.

\begin{longtabu} to \textwidth {| X[2,l] | X[3,l] | X[3,l] |}
    \caption{Porównanie Orelans i Akka.net}
    \label{tab:AkkaVsOrleans} \\
    \hline
      \multicolumn{1}{ |c| }{Cecha} 
    & \multicolumn{1}{  c| }{Orleans} 
    & \multicolumn{1}{  c| }{Akka.net} \\ 
    \hline
    \endfirsthead
    
    \multicolumn{3}{c}
    {{\bfseries \tablename\ \thetable{} -- ciąg dalszy z poprzedniej strony}} \\ 
    \hline
      \multicolumn{1}{ |c| }{Cecha} 
    & \multicolumn{1}{  c| }{Orleans} 
    & \multicolumn{1}{  c| }{Akka.net} \\ 
    \endhead

    \hline 
    \multicolumn{3}{|r|}{{Ciąg dalszy na następnej stronie}} \\ 
    \hline
    \endfoot
    
    \hline
    \endlastfoot

    Implementowany model równoległości & 
    Model Orleans może zostać przekształcony w model aktorów, gdzie każde ziarno to jeden aktor. & 
    Akka.net w całości bazuje na koncepcji modelu aktorów. \\ 
    \hline

    Transparentność położenia aktora & 
    \multicolumn{2}{ p{12cm}| }{W obu bibliotekach aktorzy nie muszą komunikować z innymi aktorami za pomocą adresów ujawniających ich fizyczną lokalizację. Zamiast tego, używają specjalnych referencji, dzięki którym nie muszą wiedzieć, gdzie znajduje się odbiorca.} \\
    \hline

    Możliwość rozproszenia systemu & 
    \multicolumn{2}{ p{12cm}| }{Zarówno Orleans jak i Akka.net zapewniają mechanizmy pozwalające na rozproszenie systemu miedzy wieloma maszynami, a także load balancing.} \\
    \hline

    Położenie aktora & 
    Determinowane przez system lub load balancer. & 
    Można narzucić adres maszyny, na której aktor będzie pracował, za pomocą odpowiedniego pliku konfiguracyjnego. \\
    \hline

    Współdzielenie pamięci & 
    \multicolumn{2}{ p{12cm}| }{Każdy z aktorów posiada własny fragment pamięci, który nie jest współdzielony z żadnym z elementów systemu aktorów.} \\
    \hline

    Przydzielanie wątków &
    \multicolumn{2}{ p{12cm}| }{Każdy aktor jest obsługiwany przez jeden wątek. Obie biblioteki nie pozwalają na wykonywanie zadań jednego aktora przez dwa równoległe wątki.} \\
    \hline 

    Zapisywanie stanu aktorów &
    \multicolumn{2}{ c| }{Oba systemy pozwalają za zapisanie stanu aktorów poza systemem.} \\
    & Stan aktorów jest zapisywany w snapshotach - zrzutach stanu ziaren.
    & Stan elementów systemu jest zapisywany za pomocą logu zdarzeń. Log pozwala też na stworzenie snapshotów. \\
    \hline 

    Serializacja wiadomości & 
    Wiadomości są zawsze serializowane. &
    Wiadomości są serializowane, gdy opuszczają lokalny system aktorów. \\
    \hline

    Tworzenie aktorów & 
    Aktorzy są tworzeni automatycznie wtedy, kiedy są potrzebni. & 
    Wymaga jawnego tworzenia aktorów. \\
    \hline

    Kontrola cyklu życia aktora & 
    Cykl życia aktora nie jest kontrolowany przez programistę. Może on jedynie podpiąć się do zdarzeń z cyklu życia. & 
    Pełna kontrola cyklu życia aktorów, włącznie z przesyłaniem wiadomości wyłączających aktorów.\\
    \hline
   
    Styl programowania & 
    Przypominający tradycyjne programowanie obiektowe z wykorzystaniem interfejsów. Na podstawie napisanego kodu generowany jest kod ziaren. & 
    Aktorzy są oddzielnymi klasami bez powiązań między sobą. Ich pisanie sprowadza się do pisania reakcji na otrzymane wiadomości.\\
    \hline

    Obsługa wyjątków & 
    Wyjątki są traktowane jako błędy w systemie. & 
    Wyjątki są traktowane jako informacja do przekazania aktorowi-rodzicowi. Od niego zależy jak zostanie potraktowany wyjątek. \\
    \hline

    Niezawodoność dostarczania wiadomości & 
    Nie zapewnia odczytu wiadomości w kolejności w jakiej przyszły.
    & Zapewnia odczyt wiadomości w kolejności w jakiej przyszły na poziomie par aktorów. \\
    & \multicolumn{2}{ c| }{Każda wiadomość jest dostarczana co najmniej raz.} \\
    \hline 
\end{longtabu}
% \end{longtable}
% \end{center}