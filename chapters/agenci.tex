\chapter{Systemy agentowe}
\section{Model aktorów}
Model aktorów to model równoległości oparty na koncepcji aktorów, którzy przesyłają między sobą wiadomości.
Model ten zakłada, że każdy z aktorów jest samodzielną jednostką posiadającą kolejkę wiadomości. Wiadomości sa przetwarzane po jednej na raz w kolejności, w której przyszły. W ramach przetwarzania mogą nastąpić zmiana stanu wewnętrznego aktora lub wysłanie jednej lub więcej wiadomości do jednego lub więcej aktorów. 

Model zakłada jak najmniejszą odpowiedzialność pojedyńczego aktora, aby umożliwić jak najwyższą równogległość i skalowalność.

\section{Przegląd rozwiązań agentowych dla środowska .NET}
\subsection{Boris.NET}
Boris jest biblioteką do tworzenia systemów agentowych stworzoną w takcie badania metod projektowania systemów agentowych w Teesside University, Wielka Brytania. Protokół Borisa pozwala na łączenie agentów napisanych za pomocą różnych języków takich jak C++, Lisp czy Java w jeden system. Boris.NET jest biblioteką napisaną przez Aliego Bojarpour do obsługi biblioteki Boris za pomocą języków \csh i \fsh. 

Elastyczność Borisa pod względem ilości obsługiwanych jest cenną cechą, gdyż pozwalałaby na pisanie adapterów i dodatkowych funkcjonalności przez różne zespoły. 
Niestety, ani Boris, ani Boris.NET nie są projektami open-source, co uniemożliwia analizę i poznanie mechanizmów wewnętrznych biblioteki. Podobnie rzecz ma się ze wsparciem społeczności, która ogranicza sie do osób ze środowiska akademickiego.

Brak też kompleksowej dokumentacji, która jest potrzebna przy poznawaniu nowej architektury oprogramowania, jaką jest podejście agentowe.

W SEKCJACH PONIŻEJ POJAWIĄ SIĘ TREŚCI NA PODSTAWIE
https://github.com/akka/akka-meta/blob/master/ComparisonWithOrleans.md
\subsection{Orleans}


\subsection{Akka.net}


\subsection{Podobieństwa między Akka.net i Orelans}


\subsection{Różnice między Akką.net i Orelans}


\subsection{Uzasadnienie wyboru Akka.net}


