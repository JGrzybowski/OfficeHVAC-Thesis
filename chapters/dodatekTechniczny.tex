\chapter{Szczegóły budowy jednostek} \label{ch:unit}

\section{Jednostka bazowa}
Abstrakcyjna klasa \lstinline{UnitActor<TInternalStatus, TParameter>} jest bazową dla wszystkich jednostek. Poniżej opisane są najważniejsze jej możliwości.

\subsection*{Inicjalizacja} \label{subsec:unitInitialization}
Każda jednostka do poprawnego działania wymaga otrzymywania sygnałów o upływie czasu. Klasy rozszerzające bazową jednostkę najczęściej będą wymagać jeszcze więcej informacji. 

Dopóki jednostki nie otrzymają początkowych wartości niektórych parametrów, takich jak czas, będą znajdować się w stanie niezainicjalizowanym i powinny jedynie odbierać wiadomości dotyczące inicjalizacji oraz zapisów na subskrypcje.
To zachowanie jest zdefiniowane w metodzie \lstinline{Uninitialized()} i może być rozszerzone o dodatkowe zachowania.

Aby zadecydować czy aktor może przejść w stan zainicjalizowany sprawdza czy odebrał wszystkie potrzebne dane za pomocą metody \lstinline{ReceivedInitialData()}. Ta metoda również jest dostępna do nadpisania i rozszerzenia o sprawdzenie dodatkowych pól.

Zachowanie jednostki w stanie zainicjalizowanym można opisać za pomocą dostępnej metody \lstinline{Initialized()}. 
W definicjach zachowań dla obu stanów należy pamiętać o zlecaniu wysyłki subskrypcji diagnostycznej przy zmianie wartości dowolnego pola wewnątrz aktora.

\subsection*{Tworzenie jednostki}
Przy tworzeniu jednostek można podać dowolną ilość adresów aktorów, do których jednostka ma się zapisać na subskrypcję. Jest to najprostszy sposób na zapewnienie, że otrzyma on wszystkie potrzebne początkowe dane. 

\subsection*{Akcje przy otrzymaniu sygnału o upływie czasu}
Jednostka zapisuje informację o ostatnim znaczniku czasowym w polu \lstinline{Timestamp}. Dodatkowo dla ułatwienia pracy przy rozszerzaniu jednostek powstały metody:
\begin{enumerate} 
    \item \lstinline{OnTimeChangedMessage(TimeChangedMessage msg)} wywoływana co przyjście wiadomości o upływie czasu.
    \item \lstinline{UpdateTime(Instant now)} metoda nadpisująca wartość ostatniego znacznika czasowego. Pozwala wykonać operacje mając do dyspozycji zarówno poprzednią jak i nową wartość znacznika. 
\end{enumerate}

\section{Jednostka cykliczna} \label{subsec:cyclicUnit}
Klasa \lstinline{CyclicUnitActor<TInternalStatus, TParameter>} to rozszerzenie jednostki bazowej z przygotowanymi mechanizmami do wykonywania pewnych operacji w odstępach czasowych. 

Metoda \lstinline{UpdateTime(Instant msg)} została rozszerzona o wyliczenie różnicy pomiędzy poprzednim a nowym znacznikiem czasowym i przekazanie jej do metody \lstinline{OnTimeUpdated(Duration timeDiff, Instant newTime)}. Pozwala to wykonywanie akcji cyklicznie co pewien okres czasu. 

Klasa zawiera jeden zestaw pola opisującego co ile (\lstinline{Threshold}) ma się wykonać pewna akcja (\lstinline{OnThresholdCrossed()}) i licznik ile czasu minęło od poprzedniego jej wykonania (\lstinline{ThresholdBuffer}). Nie ma ograniczenia ile takich zestawów będzie mieć klasa dziedzicząca po jednostce cyklicznej.

Po osiągnięciu lub przekroczeniu wartości granicznej prez licznik czasu akcja jest wykonywana tylko raz a następnie wartość licznika jest ustawiana na zero.

DIAGRAM

Drugim dodatkiem w stosunku do klasy bazowej jest dodana obsługa wiadomości dotyczączych zapisów na subskrypcję. Od implementującego zależy co będzie tą subskrypcją.

Zostało to zapisane w metodzie \lstinline{RegisterSubscribtionReceives()} która została dołączona do metod \lstinline{Uninitialized()} oraz \lstinline{Initialized()}. 

\section{Jednostka symulująca} \label{subsec:simUnit}
Niektóre z jednostek muszą mieć dostęp do aktualnego statusu pomieszczenia oraz modelu przypisanego im parametru pomieszczenia, aby wyliczyć zmianę jego wartości pomiędzy kolejnymi wywołaniami akcji. Obsługę tych wymagań dostarcza jednostka symulująca \lstinline{SimulatingUnitActor<TInternalStatus, TParameter, TParamModel>}.

\subsection*{Model parametru}
Wymaganie modelu parametru (pole \lstinline{ReceivedTemperatureModel}) zostało dodane do metody \lstinline{ReceivedInitialData()}.
Do zaktualizowania modelu parametru wykorzystywana jest metoda \lstinline{UpdateParamModel(TParamModel model)}

\subsection*{Status pomieszczenia}
Podobnie jak w przypadku modelu parametru wymaganie aktualnego statusu pomieszczenia (pole \lstinline{ReceivedInitialRoomStatus}) zostało dodane do metody \lstinline{ReceivedInitialData()}.

Za każdym razem, gdy jednostka symulująca otrzyma status pomieszczenia wywołuje metodę \lstinline{UpdateRoomStatus(IRoomStatusMessage roomStatus)}.
Jeżeli jest to pierwszy otrzymany status to dodatkowo zaraz po tym zostanie wykonana meotda \lstinline{InitializeFromRoomStatus()}. Pozwala to na m.in. wygodne przygotowanie testów implementowanych jednostek. 

\subsection{Podmiana wartości parametru}
Ze względu na aplikację symulatora oraz aby móc w przyszłości dodać możliwość ręcznego sterowania urządzeniami z systemu dodany został komunikat \lstinline{SetParameterValueMessage<TParameter>}. Pozwala on wstrzyknąć wartość parametru wewnątrz jednostki. Za wszystkie akcje dziejące się przy tym odpowiada metoda \lstinline{SetParameterValue(TParameter value)}.