\chapter{Analiza biznesowa}

\section{Model zmiany temperatury}
Przygotowanie dokładniejszych modeli parametrów pomieszczenia nie mieści się w zakresie pracy, skupiono się zatem na prostym modelu wymiany ciepła opierającym się o moc i wydajność urządzeń.

Podstawą modelu jest wzór na ciepło właściwe:
$$ c = \frac{\Delta Q}{m \Delta T} $$
Gdzie $c$ to ciepło właściwe, $\Delta Q$ to ilość energii dostarczonej do substancji, $m$ jest masą substancji, a $\Delta T$ różnicą temperatur.
Po przekształceniu we wzór na zmianę temperatury otrzymujemy:
$$ \Delta T = \frac{\Delta Q}{m \cdot c} $$
Masę powietrza możemy zamienić na iloczyn gęstości oraz objętości pomieszczenia.
$$ \Delta T = \frac{\Delta Q}{V \cdot \rho_{air} \cdot c} $$
Ilość energii dostarczonej do powietrza może być wyliczona z iloczynu mocy $P$ i czasu $\Delta t$
$$ \Delta T = \frac{P \cdot \Delta t}{V \cdot \rho_{air} \cdot c} $$

Mając taki wzór możemy zdefiniować urządzenia HVAC za pomocą ich mocy $P_{max}$ oraz zestawu trybów, w których mogą pracować.
Każdy tryb byłby opisany przez 
\begin{itemize}
    \item zużycie mocy $e$ - jaką część maksymalnej mocy używa urządzenie podczas pracy w danym trybie. 
    \item wydajność $\lambda$ - jaka część energii jest oddawana do powietrza. 
\end{itemize}
Obie wartości byłyby wyrażone wartościami z zakresu $\langle 0,1 \rangle$. Wartość efektywnej energii oblicza się wzorem
$$ \Delta Q = P_{max} \cdot e \cdot \Delta t \cdot \lambda $$

Model zakłada, że urządzenia HVAC są w stanie w jednakowym stopniu zarówno ogrzewać jak i chłodzić pomieszczenia. 

\section{Analiza wymagań systemu}
\subsection*{Możliwość rozbudowy systemu w przyszłości}
Przy okresie użytkowania budynku biurowego (średnio w Europie wynoszącym ok. 40-50 lat \cite{bib:wiekBudynku}) należy założyć, że jego właściciel będzie wymieniał w nim zarówno urządzenia HVAC jak i sensory. System musi mieć zatem strukturę pozwalającą na późniejszejszą rozbudowę i ulepszanie. 

Jeżeli właściciel budynku postanowi unowocześnić budynek i wymienić sensory na nowsze np. odczytujące dodatkowo poziom wilgotności w pomieszczeniu, system powinien mieć możliwość wykorzystać dodatkowe informacje.

Dodanie nowych, dynamicznie zmieniających się parametrów, takich jak wilgotność czy nasłonecznienie, pozwalałoby udoskonalić model temperatur i zbliżyć go do rzeczywistego modelu fizycznego. Dokładniejszy model pozwala na lepszą predykcję, a co za tym idzie większe oszczędności.

Jednocześnie, dodanie mniej dynamicznych parametrów, na które jesteśmy w stanie wpłynąć dawałoby możliwość sprawdzenia i skorygowania ich. Przykładowo poziom tlenu pozwala ocenić czy osobom w pomieszczeniu nie jest zbyt duszno i przewietrzyć pomieszczenia za pomocą dostępnych urządzeń. 

% Oczywiście, każde dodanie takiego parametru wiązałoby się z nowym typem sensora. O ile badane zajwisko zmianiałoby się na tyle wolno, że moglibyśmy na nie oddziałować, byłoby możliwe dodanie aktuatora.

\subsubsection*{Zaawansowana optymalizacja kosztów}
Właściciel budynku może uznać, że oszczędności wypracowane przez system nie są już wystarczające. 
Dokładniejszy matematyczny model pozwoliłby dokładniej szacować zużycie energii elektrycznej i bardziej zoptymalizować wydatki. 

Na podstawie wiedzy ekspertów można wypracować bardziej zaawansowany model dopasowany do ilości informacji dostarczanej przez sensory i najemców.
Możliwość parametryzacji modeli pozwoliłaby dostrajać modele pod konkretne budynki i poprawić precyzję predykcji.

\subsection*{Wygoda aktualizacji systemu}
Wyłączenie systemu klimatyzatorów może być uciążliwe dla osób korzystających z budynku. 
Serwisowanie i obsługa systemu zarządzania urządzeniami powinna jak najmniej wpływać na komfort użytkowników.

Zdalna podmiana modeli umożliwiłaby aktualizację parametrów modelu lub podmianę na inaczej skonstruowany model (np. przewidujący temperatury z innych parametrów pomieszczenia) bez wyłączania systemu, czyli i bez utraty komfortu dla najemcy. 

\subsection*{Zróżnicowanie źródeł danych o wykorzystaniu pomieszczeń}
W różnych firmach używa się różnych systemów do ustalania terminów spotkań czy rezerwacji sal np. Microsoft Exchange Server czy IBM Domino.
System powinien pozwalać na przekazywanie informacji o wykorzystaniu pomieszczeń systemu HVAC na różne sposoby w zależności od najemcy.

\subsection*{Diagnostyka urządzeń i systemu}
Administracji budynku może zależeć na możliwości monotorowania stanu systemu, aby np. wykryć wadliwy sensor lub przyznać dodatkowy rabat najemcy za odpowiednią ilość oszczędzonej energii.

Dla celów diagnostycznych potrzebny jest łatwy dostęp do danych o stanie urządzeń oraz danych z samego systemu HVAC. Musi zatem istnieć sposób, aby w łatwy sposób połączyć się z systemem HVAC i zerbać informacje o jego stanie.
